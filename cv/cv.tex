% From: http://nitens.org/taraborelli/cvtex
% Some rights reserved: http://creativecommons.org/licenses/by-sa/3.0/

%!TEX TS-program = xelatex
%!TEX encoding = UTF-8 Unicode

\documentclass[10pt, a4paper]{article}
\usepackage{fontspec}
\usepackage{verbatim,placeins}

% DOCUMENT LAYOUT
\usepackage{geometry}
\geometry{a4paper, textwidth=5.5in, textheight=9in, marginparsep=7pt, marginparwidth=.6in}
\setlength\parindent{0in}
\usepackage{hanging}

% FONTS
\usepackage[usenames,dvipsnames]{xcolor}
\usepackage{xunicode}
\usepackage{xltxtra}
\defaultfontfeatures{Mapping=tex-text}
\setromanfont [Ligatures={Common}, Numbers={OldStyle}, Variant=01]{Linux Libertine O}
\setmonofont[Scale=0.7]{Monaco}

% CUSTOM COMMANDS
\newcommand{\amper}{{\fontspec[Scale=.95]{Linux Libertine O}\selectfont}}
\newcommand{\html}[1]{\href{#1}{\scriptsize\textsc{[html]}}}
\newcommand{\pdf}[1]{\href{#1}{\scriptsize\textsc{[pdf]}}}

\usepackage{marginnote}
\newcommand{\years}[1]{\marginnote{\scriptsize #1}}
\renewcommand*{\raggedleftmarginnote}{}
\setlength{\marginparsep}{7pt}
\reversemarginpar

% HEADINGS
\usepackage{sectsty}
\usepackage[normalem]{ulem}
\sectionfont{\mdseries\upshape\Large}
\subsectionfont{\mdseries\scshape\normalsize}
\subsubsectionfont{\mdseries\upshape\large}

% PDF SETUP
\usepackage[xetex, bookmarks, colorlinks, breaklinks, pdftitle={Matt Jibson - vita},pdfauthor={Matt Jibson}]{hyperref}
\hypersetup{linkcolor=blue,citecolor=blue,filecolor=black,urlcolor=MidnightBlue}

\begin{document}
\reversemarginpar
{{\huge Matt Jibson}\\[1.5cm]
\parbox{.5\linewidth}{
1908 Sage Drive\\
Golden, CO 80401
}
\parbox{.5\linewidth}{
\textsc{c}\quad\texttt{303-902-6948}\\
{\fontspec{Arial Unicode MS}{✉}}\hspace{.26cm}\href{mailto:matt.jibson@gmail.com}{matt.jibson@gmail.com}\\
\textsc{w}\hspace{.24cm}\href{http://mattjibson.com/}{http://mattjibson.com/}\\
}

\subsubsection*{Areas of interest}
Digital synthesis of pipe organs and pianos. I have written a paper on a method to synthesize pipe organs, referenced below. This research still has open problems.

\section*{Education}

\noindent\years{2009}\textbf{M.S., Electrical Engineering}\\
\emph{Colorado State University} \\
{\small \textsc{advisor:} Tom Chen.}\\[.2cm]
\years{2007}\textbf{B.S., Computer Engineering}\\
\emph{Colorado State University.}\\[.2cm]
\years{2007}\textbf{B.M., Piano Performance}\\
\emph{Colorado State University.}

\section*{Awards}

\years{2007}2$^{nd}$ place in the Colorado State University Senior Design E-days Awards in Electrical Engineering. A dual electronic \amper{} pipe organ was built.\\[.1cm]
\years{2006}Wendel Diebel Award from the Colorado State University Music Department for outstanding musicianship.

%\section*{IT \amper{} programming skills}
%Markup languages (XHTML, CSS, XML, JSON).\\[.1cm]
%Scripting languages (PHP, Python, shell script, JavaScript).\\[.1cm]
%Compiled languages (C, C++, Java).\\[.1cm]
%Query languages (SQL).\\[.1cm]
%Data analysis (MATLAB).\\[.1cm]
%Revision control (Git, CVS, Subversion).\\[.1cm]
%Digital typesetting ({\fontspec{Times New Roman}\LaTeX}).

\section*{Publications}

\subsection*{1. Unpublished works}
\begin{hangparas}{.25in}{1}
\textbf{Jibson, M.W.}, \emph{Electrochemical Biosensor Array Characterization}, M.S. Thesis in Electrical Engineering, Colorado State University, May 2007.\\

\textbf{Jibson, M.W.}, \emph{Organ Sound Synthesis by Harmonic Interpolation}, 2009.\pdf{http://mjibson.github.com/pubs/schalmei/schalmei.pdf}\\
\end{hangparas}

\subsection*{2. Software}
\begin{hangparas}{.25in}{1}
Jibson, R.W., Rathje, E.M., \textbf{Jibson, M.W.}, and Lee, Y.W., in press, SLAMMER---Seismic LAndslide Movement Modeled using Earthquake Records, \emph{U.S. Geological Survey Techniques and Methods}, on CD-ROM and Internet.\\

Jibson, R.W., and \textbf{Jibson, M.W.}, 2003, Java programs for using Newmark’s method and simplified decoupled analysis to model slope performance during earthquakes, \emph{U.S. Geological Survey Open-File Report 03-005}, on CD-ROM and Internet.\\

Jibson, R.W., and \textbf{Jibson, M.W.}, 2002, Java programs for using Newmark’s method to model slope performance during earthquakes, \emph{U.S. Geological Survey Open-File Report 02-201}, on CD-ROM.\\

Jibson, R.W., and \textbf{Jibson, M.W.}, 2001, Programs for using Newmark’s method to model slope performance during earthquakes, \emph{U.S. Geological Survey Open-File Report 01-116}, on CD-ROM.
\end{hangparas}

\vfill{}
\begin{center}
{\scriptsize Last updated: \today\- •\- Typeset in \href{http://nitens.org/taraborelli/latex}{{\fontspec{Times New Roman}\XeTeX}}\\\href{http://mattjibson.com/cv.pdf}{http://mattjibson.com/cv.pdf}}
\end{center}

\end{document}
